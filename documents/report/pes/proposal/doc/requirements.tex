As stated previously, some work still has to be done in order to get the basic functionality running. Therefor it's crucial that the hardware will be upgraded with new PCB's and components, the control loop implemented and visualization on the screen to be updated to reflect the social media interaction better. Furthermore, no communication exists yet between the two microcontrollers. A protocol has to be researched, and implemented. All these tasks are required for the xmega. Another part of the project will be the programming of the ESP8266, which will involve more time. First off, a programming environment and language will have to be selected for the ESP8266, as there are many different languages and frameworks to program it in. As soon a suitable framework has been found, implementation of the required functions can begin. The ESP8266 should be able to communicate with the xmega, be able to act as an access point when it can't connect to another access point, receive the current time from the Internet, download emails from the user and if time allows integrate with social media platforms. Below in fig. \ref{fig:block} a small diagram can be seen of what will have to be realized in order to get a working first version.
\begin{figure}[H]
	\includegraphics[width=\textwidth]{./fig/PES.png}
	\caption{Rough sketch of the requirements.}
	\label{fig:block}
\end{figure}
The application part will activate one or multiple sub-systems such as the weather/calendar or time sub system. If either of those systems require an update, they will create a new net request. This request will be a data structure and will be passed to the ESP8266 driver. Here the data structure will be translated into the physical layer and sent via UART. Once this arrives at the ESP8266 it will have to be interpreted, and if needed data will be sent and or received from the radio. The radio will be in contact with microservices that are ran on a SaaS solution where it will be able to collect data from facebook/email/twitter. Other restful requests will be made to acquire the newest time and the latest weather.\\ Alongside this the ESP8266 will also act as an endpoint and access point. So that users will be able to set up a connection on the ESP8266 itself via a default route to a webpage hosted on the ESP8266.\\
Research has to be conducted on the requirements of getting information of one of theses social media sites. As well as the encryption that goes with it. How the got the most accurate time from the internet. To detect a location from a local network so the user won't have to select that itself. As well as how to generate restful requests and how to handle HTTPS traffic. Furthermore a webserver will have to be constructed on the ESP8266, and a flexible communication protocol between the two chips will have to be selected and implemented. Lastly a well organized structure should be kept on the created operating system.
