The whole board design has been made in \textit{KiCAD}. \textit{Git} was used for version control. In the schematics (Appendix \ref{append:power}) one can see that the whole board consists of three main building blocks: Connectors, a LED driver with feedback and two step-down converters. A part of the LED driver is also "abused" to drive the OLED. 
\subsection{Microcontroller Power Supply}
There are two step-down converter ICs on the power board. It is the LM2840 which in combination with a simple voltage divider ensures the 2.8 V for Vcc. All step-down converters use the same inductor with a value of 33 uH. It is a low-cost, quite small, shielded inductor which is ment to be used for switching power supplies. Moreover all step-down converters are enhanced with a SMD schottky diode and a of course SMD capacitor for smoothing the output signal. As it is good practice to do so all ICs are making use of decoupling capacitors. 
\subsection{Designated USB Charging Port}
For charging ones phone the TS30012, another step-down converter IC, is used. The feedback voltage divider of this IC is already onboard and does not need to be provided externally as the IC provides fixed 5 V. The output is connected to a USB connector type A. This IC can deliver up to 2 A. An interesting feature of the phone charging circuitry on the power board is the "Dedicated Charging Port" (DCP) functionality. The TPS2514 is a small, easy-to-use, 6-pin component, which complies to the USB standard and a majority of the minefield of propriatary standards to signal a DCP. What does that mean? Well this means, that if you connect your IPhone, it will know, that it can draw more than 100 mA, which are the minimum provided by a normal USB port. Otherwise the current drawn by the phone will be limited. The charging functionality can be turned on and off via a GPIO pin. The TS30012 comes in a QFN16 package (pad pitch of 0.5 mm) to save space.   
\subsection{HW Debugging}
For testing purposes a lot of test points have been created on the board. Futhermore there are  LEDs for different voltages.
\subsection{RGB LED Driver}
The LED driver consists of an actual power electronics part and a feedback part. The idea is, that the voltage driven through the three color channels of the RGB LED can be controlled by software (PID controller). In the power electronics part there are three analog circuits mainly consisting out of a p-channel MOSFET, which is switched by a NPN bipolar transistor. This bipolar transistor gets its intput signal from the µProcessor (PWM). By pulling the 20 V to GND the PMOS "sees" a negative Vgs and opens.             
