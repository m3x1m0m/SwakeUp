\chapter{Introduction} 
\label{chap:introduction} 
It is a well-known fact, that Sweden is a country with big variations of the
daily hours of sun throughout the year. Far up north the sun does not set
anymore in January. Even in Stockholm in January, the most dark month of the
year, the sun rises at 8:47 am and sets at 2:55 pm
(cf.\cite{other:visitsweden}). According to \myemph{Sveriges Radio} "many Swedes
suffer from the winter blues or seasonal affective disorder" (cf.
\cite{other:sverigesradio}). In a strong winter every source of light is a
source of happiness. This is why a wakeup light, which is based on a strong
light source (at least 10 W RGB LED), is able to give one the optimal start into
a dark winter day with an artificial sunrise as bright as a real sun shining
through the window.  
\newpar
This report describes the Swakeup (from engl. ”Swedish Wakeup Light”), a device
communicating to the user not only through light. It does not simply wake one
up, but also gives one information about social media, latest mails, calendar
and weather. The user interface consits besides of a high-power LED of an OLED
screen. Swakeup is also part of the IoT as it has the ability to communicate via
IEEE 802.11. This of course enables a lot of possibilities e.g. connecting your
phone to the wakeup light. A lot of effort has been put into the designing
maxim, that everything should be as small as possible. The whole electronics fit
on an base area of 5 cm x 4 cm. So the Swakeup fits smoothly on the bedside
table. And honestly: What is the last thing people are doing before they go to
sleep? Right! They look on your phone. That is why Swakeup comes with a USB
charger for your e.g. phone as well. Another design maxim of this product is
cheapness. Everybody should be able to buy one. As all engineering work is
available online, it gives people (with the corresponding knowledge) the
opportunity to build a wakeup light themselves, look up what this device is
doing with their personal data or even contribute to the product.
