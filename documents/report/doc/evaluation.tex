\chapter{Evaluation}
\label{chap:evaluation}
\section{Hardware}
The hardware right now is in revision 2.0. Revision 3.0 is currently under
developemnet. Revision 3.0 will be the third recision. The current revision
,where all the software is running on is revision 2.0. It has been fabricated in
January 2017 and the design process (including revision 1.0) began in november
2016. There were severe mistakes on the first hardware revision, which were not
fixable so easily, especially because of the fact, that everything is quite
small. In the second hardware revision there are still smaller mistakes.
Moreover upgrades shall be taken into account for the revision 3.0. 

\subsection{Issues}
\label{subsec:issues}
The feedback mechanism, which is crucial of course for any kind of controll loop
does not work in the case of the RGB channels. This is because the LM324
quadruple operational amplifier is connected to a single power supply. In this
case it is necessary to connect the shunt resistors according to the current
flow direction to the differential amplifier. This has been confused and needs
to be fixed. 
\newpar
The supply voltage of \SI{2.8}{\volt} does not seem to be longer suitable for the
application. The system crashes, when the supply voltage is below approximately
\SI{3.1}{\volt}. The easy fix for the next revision is a supply voltage of
\SI{3.3}{V}. Further investigations have to be made here. 
\newpar
The used coils have to be checked once more if the current rating fits. The coil
used for the USB charger is definitely not suitable. It catches flames, when the a current above \SI{600}{\milli\ampere} is drawn from the charging prot. 

\subsection{Upgrades}
\label{subsec:upgrades}
A few low-pass filters shall be moved from the power board to the logic board
(saves space). 
\newpar
A ADC converter channel has to be tied to GND for calibration,
which can be traced back to some special properties of the 12 bit ADC of the
Xmega. 
\newpar
Also a buzzer shall be part of the next revision to ensure, that the user
wakes up.
\newpar
Since the RGB LEDs are quite cheap and the driver circuitry turned
out to be pretty powerfull, more LEDs (2-4 in total) shall be added to the system. 
\newpar
A housing is an essential part of such a product. That is why a small box will
be constructed, which fits both boards, a couple of LEDs and the screen. 

\section{Software}

