\chapter{Evaluation}
\label{chap:evaluation}
\section{Hardware}
\label{sec:evaluation_hw}
The hardware right now is in revision 2.0. Revision 3.0 is currently under
developemnet. Revision 3.0 will be the third recision. The current revision
,where all the software is running on is revision 2.0. It has been fabricated in
January 2017 and the design process (including revision 1.0) began in november
2016. There were severe mistakes on the first hardware revision, which were not
fixable so easily, especially because of the fact, that everything is quite
small. In the second hardware revision there are still smaller mistakes.
Moreover upgrades shall be taken into account for the revision 3.0. 
\newpar
Table \ref{tab:hw_status_quo} gives an overview about the status quo of the hardware.
\begin{table}[H]
	\centering
	\begin{tabularx}{\textwidth}{llX}
		\textbf{HW Block}  	& \textbf{Working}& \textbf{Problem}  	\\\hline
		USB Charging 	& \checkmark	& 				\\
		Screen Driver  	& \checkmark    & 				\\
		Screen Driver Feedback& \checkmark    & 				\\
		Vcc for µC   	& \checkmark    & 				\\
		ESP   		& \checkmark    &                             	\\
		Terminal 	& \checkmark	&				\\
		LED Driver	& \checkmark	& 				\\
		LED Driver Feedback & & Reversed inputs of the differential ampifier \\
		USB DCP		& \checkmark	& 				\\
		Reverse Polarity Protection & \checkmark & \\
		\hline
	\end{tabularx}
	\caption{Status Quo Hardware}
	\label{tab:hw_status_quo}
\end{table}

\subsection{Issues}
\label{subsec:issues_hw}
The feedback mechanism, which is crucial of course for any kind of controll loop
does not work in the case of the RGB channels. This is because the LM324
quadruple operational amplifier is connected to a single power supply. In this
case it is necessary to connect the shunt resistors according to the current
flow direction to the differential amplifier. This has been confused and needs
to be fixed. 
\newpar
The supply voltage of \SI{2.8}{\volt} does not seem to be longer suitable for the
application. The system crashes, when the supply voltage is below approximately
\SI{3.1}{\volt}. The easy fix for the next revision is a supply voltage of
\SI{3.3}{V}. Further investigations have to be made here. 
\newpar
The used coils have to be checked once more if the current rating fits. The coil
used for the USB charger is definitely not suitable. It catches flames, when a current above \SI{600}{\milli\ampere} is drawn from the charging prot. 

\subsection{Upgrades}
\label{subsec:upgrades_hw}
A few low-pass filters shall be moved from the power board to the logic board
(saves space). 
\newpar
A ADC converter channel has to be tied to GND for calibration,
which can be traced back to some special properties of the 12 bit ADC of the
Xmega. 
\newpar
Also a buzzer shall be part of the next revision to ensure, that the user
wakes up.
\newpar
Since the RGB LEDs are quite cheap and the driver circuitry turned
out to be pretty powerfull, more LEDs (2-4 in total) shall be added to the system. 
\newpar
A housing is an essential part of such a product. That is why a small box will
be constructed, which fits both boards, a couple of LEDs and the screen. 

\section{Software}
\label{sec:evaluation_sw}
The software is working and almost finished on all different levels. One could
discuss about the basic principles of the operating system implemented. For
instance if a timer triggers an event every \SI{10}{\milli\second} the listeners
i.e. the tasks, which are supposed to be executed with this period will
experience a heavy jitter. This jitter is dependent on how many listeners this
event currently has and how many other events have to be handeled by the kernel at the same time. If a task implemented takes too much time to execute or executes to
frequently, the system crashes. One has to be carefull with e.g. PID controllers
as those can generate heavy workload, which can not be handeled anymore by the
ressources the Xmega provides. The operating system does in that sense not
provide real-time functionality.
\newpar
Table \ref{tab:sw_status_quo} deals with an overview of the software status quo.
\begin{table}[H]
	\centering
	\begin{tabularx}{\textwidth}{llX}
		\textbf{SW Block}  	& \textbf{Working}& \textbf{Problem}
		\\\hline
		UART 		& \checkmark	&
		\\
		SPI	  	& \checkmark    &
		\\
		EPROM   	& \checkmark    &
		\\
		Timer   	& \checkmark    &
		\\	
		ESP8266		& \checkmark	&
		\\
		Terminal	& \checkmark	&
		\\
		SEP525F		& \checkmark	&
		\\
		Wifi		& \checkmark 	&
		\\
		Command		& \checkmark	&
		\\
		Log		& \checkmark	&
		\\
		Screen		& \checkmark	&
		\\
		Timekeeper	& \checkmark	&
		\\
		Controller	&		&
		\\
		Core		& \checkmark	&
		\\
		Weather		& \checkmark	&
		\\
		Clock		& \checkmark	&
		\\
		Social		& 		& Not implemented yet
		\\
		PWM & \checkmark & \\
		ADC & \checkmark & \\
		Lightcontroller & \checkmark & \\
		PID & \checkmark & \\
		Advanced Light Patterns &  & Debugging still ongoing\\
		Play 8 bit sounds & & Not implemented, HW missing\\
		\hline
	\end{tabularx}
	\caption{Status Quo Software}
	\label{tab:sw_status_quo}
\end{table}

\subsection{Issues}
It is crucial to make the system safer. Through a software malfunction it is
quite easy to burn LEDs or the screen. This needs to be avoided. Time needs to
be invested into the developement of safety mechanisms to avoid the destruction
of the hardware.
\newpar
The code size needs to be optimized. More than 80 \% of the flash of the Xmega
is currently used. There is not too much space anymore for upgrades. 
\newpar
In the end four PID controllers shall run parallely on the platform. The
facilities for debugging and tuning a PID controller are give, but the fine
tuning of \ensuremath{K_P}, \ensuremath{K_I} and
\ensuremath{K_D} still has to be done for the four different controllers. 
\subsection{Upgrades}
The software upgrades mainly orientate on the hardware upgrades, which will be
made with hardware revision 3 (cf. \ref{subsec:upgrades_hw}).
