It is a well-known fact, that it is quite dark in Sweden in the winter. In a strong winter every source of light is a source of happiness. This wakeup light, which is based on a strong light source (10 W RGB LED), is able to give one the optimal start into a dark winter day. The \textit{Swakeup} (from engl. "Swedish Wakeup Light") is communicating to the user through the light. It does not simply wake one up, but also gives one information about Facebook, latest mails, calendar and weather. The user interface consits besides of a big LED of an OLED screen. \textit{Swakeup} is also part of the \textit{IoT} as it has the ability to communicate via \textit{IEEE 802.11}. This of course enables a lot of possibilities e.g. connecting your phone to the wakeup light. A lot of effort has been put into the designing maxim, that everything should be as small as possible. The whole electronics fit on an base area of 5 cm x 4 cm. So the \textit{Swakeup} fits smoothly on the bedside table. And honestly: What is the last thing you are doing before you go to sleep? Right! You look on your phone. That is why \textit{Swakeup} comes with a USB charger for your e.g. phone as well.\\
\subparagraph{Current state}
As this project is a continuation from a project of another course, certain steps have already been made. A first version of hardware is developed, containing hardware mistakes. The hardware consists out of 2 processors, an energy efficient Atmel xmega, taking care of the control loop and acts as interface for the screen. And an ESP8266 which will act as a gateway to the internet. No work has been done on the ESP8266 side yet, neither has communication been set up. The control loop that should take care of setting the LED to the right brightness is also not been implemented yet. 
